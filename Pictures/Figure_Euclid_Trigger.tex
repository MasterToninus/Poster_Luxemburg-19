%+------------------------------------------------------------------------+
%| Diagram: How to trigger the HCMM costruction in R3
%| Author: Antonio miti
%+------------------------------------------------------------------------+

\documentclass[border=10pt, tikz]{standalone}
\usepackage{tikz-cd}
\usepackage{mathtools}
\usepackage{amsfonts}
\usetikzlibrary{shapes.geometric,fit}
\usetikzlibrary{decorations.pathmorphing}
\tikzset{
   dashsquare/.style={rectangle,draw,dashed,inner sep=0pt,blue,fit={#1}}
}


\begin{document}
\begin{tikzcd}[column sep=large,
   execute at end picture={
     \node[label=below:{\tiny CE complex},dashsquare=(L1)(L2)]{};
     \node[label=below:{\tiny Observables},dashsquare=(R1)(R2)]{};
		}]
	&	& \Omega^3(M) \ar[ddd,leftrightarrow,green!60!black,bend left=60,"\ast"]\\
	& \mathfrak{X}(M) \ar[r,red,"\alpha"] \ar[dr,blue,"\flat",shift left=1ex]\ar[dr,blue,leftarrow,"\sharp"',shift right=1ex]& \Omega^2(M) \ar[u,"d"']  \ar[d,leftrightarrow,green!60!black,bend left=60,"\ast"] \\
	|[alias=L1]| \mathfrak{g} \ar[ru,hookrightarrow,"V_{\cdot}"] \ar[rr,purple,"f_1"]
	& 
	& |[alias=R1]| \Omega^1_{(ham)}(M) \ar[u,"d"'] \\
	|[alias=L2]| \mathfrak{g} \wedge \mathfrak{g}  \ar[u,"\partial"]\ar[rr,purple,"f_2"]
	& \color{purple}\underbrace{\qquad}_{HCMM} 
	& |[alias=R2]| \Omega^0(M) \ar[u,"d"']
\end{tikzcd}
\end{document}

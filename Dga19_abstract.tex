%+----------------------------------------------------------------------------+
%| Poster: relevant background of my research project "Differential geometry and applications to modern physics"
%| Author: Antonio miti
%| Event: Differential Geometry and applications 2019
%| Place: Hradec Králové, Cech Republic
%| 
%+------------------------------------------------------------------------+

\documentclass[11pt,a4paper,twoside]{article}

\usepackage{fancyhdr}
%\usepackage[T1]{fontenc}
\usepackage{amsfonts}


%-----------  SIZE  ------------%
\parskip=0.5ex
\oddsidemargin= 0.35cm
\evensidemargin= 0.35cm

\parindent=1.5em
\textheight=23.0cm
\textwidth=15.5cm


\begin{document}

\title{Multisymplectic manifolds and Homotopy co-momentum maps}
\author{Antonio Michele Miti\\ UCSC Brescia \& KU Leuven}

\date{September ?th, 2019}
\maketitle


\begin{abstract}

This poster is intended as a short survey of part of the background material on which is based my Ph.D. research project.
The aim is to give a schematic account on $n$-plectic structures (also called \emph{multisymplectic})  and the generalization of moment maps to this setting called homotopy co-momentum map.
The key point will be to shift our attention from points to observables recognizing that the latter are encoded in an L-$\infty$ algebra structure which is a higher analogue of a Lie algebra where one requires the Jacobi identity to be satisfied up to homotopies.
Some concrete examples inspired by geometric mechanics will also be described.
 
\end{abstract}




%------------------------------------------------------------------------------------------------
% Bibliography (BibTex)
% https://arxiv.org/hypertex/bibstyles/
%------------------------------------------------------------------------------------------------
			%\nocite{*}
			\bibliographystyle{ieeetr}
			\bibliography{biblio}
%------------------------------------------------------------------------------------------------

\end{document}



%+----------------------------------------------------------------------------+
%| Poster: relevant background of my research project "Differential geometry and applications to modern physics"
%| Author: Antonio miti
%| Event: Supergeometry, supersymmetry and quantization
%| Place: University of Luxembourg
%| Date: December 16-19, 2019
%| Template Credits: https://www.latextemplates.com/template/jacobs-landscape-poster
%| 
%+----------------------------------------------------------------------------+



%----------------------------------------------------------------------------------------
%	PACKAGES AND OTHER DOCUMENT CONFIGURATIONS
%----------------------------------------------------------------------------------------
\documentclass[final,a0paper,20pt,
						pdftex,
            pdfauthor={Antonio Michele Miti},
            pdftitle={Homotopy title},
            pdfsubject={Poster for the conference Super19 in Luxemburg},
            pdfkeywords={Some Keywords},
            pdfproducer={Latex with hyperref, or other system},
            pdfcreator={pdflatex, or other tool}
            ]{beamer}

\usepackage{amsmath,amssymb,amsfonts}



\usepackage[orientation=portrait, size=a0, scale=1.2]{beamerposter}
\usetheme{Super19} % Use the confposter theme supplied with this template
\usepackage{graphicx} 
\usepackage{marvosym, fourier-orns, fontawesome} %funny icons
\usepackage[mode=buildnew,subpreambles]{standalone}
\usepackage{tikz,tikz-cd}
\usepackage{caption}
\usepackage{empheq}

%-----------------------------------------------------------
% Some shortcut
%-----------------------------------------------------------
\newcommand{\pinned}[1]{
    \faThumbTack 
    \hfill
    %\hspace*{\fill} 
    #1
    \hfill \faThumbTack
    \\
    }

%----------------------------------------------------------------------------------------
%	TITLE SECTION 
%----------------------------------------------------------------------------------------

\title{Multisymplectic manifolds and Homotopy co-momentum maps} 
\author{Antonio Michele Miti}
\institute{	
				Department of Mathematics and Physics, 	Universit\`{a} Cattolica del Sacro Cuore, Brescia, Italy
				\\
				Department of Mathematics, KU Leuven, Leuven, Belgium
				}
%----------------------------------------------------------------------------------------

\begin{document}

\addtobeamertemplate{block end}{}{\vspace*{2ex}} % White space under blocks
\addtobeamertemplate{block alerted end}{}{\vspace*{2ex}} % White space under highlighted (alert) blocks

\setlength{\belowcaptionskip}{2ex} % White space under figures
\setlength\belowdisplayshortskip{2ex} % White space under equations

%*************************************************************************
\iffalse			\section{ DEBRIS }					\fi%*******************************
%*************************************************************************
\begin{frame}[t,shrink]
\begin{columns}
\begin{column}{\sepwidexternal}\end{column} % Empty spacer column
	\begin{column}{\onecolwid}

		\begin{block}{cose}
			If the infinitesimal action $v$ is multisymplectic with respect to $\omega$ and $\tilde{\omega}$, 
			it admits a HCMM $(f):\mathfrak{g}\to L_\infty(M,\omega)$ with respect to $\omega$,
			and $B$ is strictly conserved,
			then the $L_\infty$-morphism $(\tilde{f}):\mathfrak{g}\to L_\infty(M,\tilde{\omega})$
			, given in components by
			Bluff!!!
			\begin{displaymath}
				\varphi_{k+1} =  \frac{B_k}{c_k} 
				\underbrace{\langle\cdot,\cdot\rangle_-\circ\dots\circ\langle\cdot,\cdot\rangle_-}_{k ~\text{times}}
			\end{displaymath}
		\end{block}
	
		\begin{block}{altre cose}
			\url{https://commons.wikimedia.org/wiki/File:Displacement_of_a_continuum.svg}

			\center \danger Several ways to introduce this concept... \danger
			\includestandalone[width=0.6\textwidth]{Pictures/Linfinity_keywords}	
		\end{block}	
	\end{column}
	%----------------------------------------------------------------------------------------
	\begin{column}{\sepwidinternal}\end{column} % Empty spacer column	
	\begin{column}{\onecolwid}

		%+---------------------------------+
		%| Contacts                        |
		%+---------------------------------+
		\setbeamercolor{block alerted title}{fg=black,bg=norange} % Change the alert block title colors
		\setbeamercolor{block alerted body}{fg=black,bg=white} % Change the alert block body colors
		\vspace{1cm}
		\begin{alertblock}{Contact Information}
		\small
		\begin{center}
			\begin{tabular}{c c c c}
				Contacts\quad: 
				&\ComputerMouse $\;$ 
				&  \href{https://dmf.unicatt.it/miti/}{https://dmf.unicatt.it/miti/} 
				& $\;$ \ComputerMouse 
				\\
				&\Letter $\;$ 
				& \href{mailto:am.miti@dmf.unicatt.it}{am.miti@dmf.unicatt.it} 
				& $\;$ \Letter
			\end{tabular}		
		\end{center}
			Contacts: \qquad \href{https://dmf.unicatt.it/miti/}{\Mundus~ https://dmf.unicatt.it/miti/} \qquad 
			\href{mailto:am.miti@dmf.unicatt.it}{\Letter~ am.miti@dmf.unicatt.it}.
		\end{alertblock}

		%+---------------------------------+
		%| Bibliography                    |
		%+---------------------------------+
		\setbeamercolor{block alerted title}{fg=black,bg=norange} % Change the alert block title colors
		\setbeamercolor{block alerted body}{fg=black,bg=white} % Change the alert block body colors
		\vspace{1cm}
		\begin{alertblock}{Bibliography}
			\nocite{*}
			\bibliographystyle{plain}
			\footnotesize
			\bibliography{biblio}
		\end{alertblock}
		
		\begin{block}{Goals}
			Multisymplectic manifolds, $L_\infty$ observables and homotopy co-moment maps are an higher generalization of symplectic manifolds, Poisson algebras and co-moment maps respectively.
			\\
			Being the latter a particularly subtle and technical concept, we worked out some new meaningful examples of possible interests in different fields of mathematics.
			
			\flushright
			\begin{tabular}{p{0.25\linewidth}p{0.4\linewidth}p{0.25\linewidth}}
				\faMapSigns Contents:
				& \rotatebox[origin=c]{90}{\faMailReply} 
				Relevant background
				& New results \reflectbox{\rotatebox[origin=c]{90}{\faMailReplyAll}}
			\end{tabular}		
		
		\end{block}

		\begin{block}{Icons}
			Awesome:\\
			\faGlobe
			\faAnchor
			\faEnvelope
			\faEnvelopeO
			\faEnvelopeSquare
			\faMailForward
			\faMailReply
			\faMailReplyAll
			\faPaperclip
			\faRotateLeft
			\faRotateRight
			\faThumbTack
			\faLevelDown
			\faArrowDown
			\faPhone
			
			Marvosym:\\
			\ComputerMouse \SerialInterface \Keyboard
			\MVAt \Mundus
			
			Fourier:\\
			\noway 
			\danger
			\textxswup \textxswdown
			\lefthand \righthand \decosix \bomb
		
			Rotation:\\
			\rotatebox[origin=c]{90}{\faMailReply}
			\reflectbox{\rotatebox[origin=c]{90}{\faMailReplyAll}}
		\end{block}


\end{column} % End of column 2.1
\begin{column}{\sepwidexternal}\end{column} % Empty spacer column
\end{columns} % End of all the columns in the poster
\end{frame}		


consider two $B$-related $n$-plectic forms. There exists a bundle morphism
\begin{displaymath}
{
	\begin{tikzcd}[column sep= small,row sep=0ex,
				/tikz/column 1/.append style={anchor=base east}]
		\tau_B \colon &
		E^n = TM \oplus \Lambda^{n-1} T^\ast M \ar[r]& E^n ~, \\
	 &\binom{X}{\alpha} \ar[r, mapsto] & 
	 \binom{X}{\alpha - \iota_X B} =
 \binom{X}{\alpha} - \binom{0}{\iota_X B}
	\end{tikzcd}		
}
	 ~,		
\end{displaymath}
from the corresponding standard Vinogradov algebroid into itself

					Consider $\alpha_i \in \Omega_{\text{Ham}}^{n-1}$,
					$\gamma_i \in \bigoplus_{i=0}^{n-2}\Omega^n$
					Denote $ x_i = v_{\alpha_i}$, $e_i = \alpha_i + \gamma_i$

					Then	
					$$\tilde{f}_k = (f_k +b_k)$$
					with
					$$
					b_k(e_1,\dots,e_k) 
					= - \varsigma(k+1)\iota_{v_\cdot} B
					= - (-)^k \iota_{x_1}\dots\iota_{x_1} B
					$$	


\end{document}


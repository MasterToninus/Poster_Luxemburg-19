%+------------------------------------------------------------------------------------------------------------------------+
%| Abstract: Poster on my research project "Differential geometry and applications to modern physics"		|
%| Author: Antonio Michele miti
%| Event: Supergeometry, supersymmetry and quantization
%| Place: University of Luxembourg
%| Date:	December 16-19, 2019, 
%| Note: This poster was intended as a short survey on the main results
%|			of my doctoral research project, together with some background 
%|			material.
%|			The first aim is to give a schematic account on $n$-plectic structures (also called \emph{multisymplectic}) and the generalization of moment maps to this setting called \emph{homotopy co-momentum map}.
%| 
%+-----------------------------------------------------------------------------------------------------------------------+

\documentclass[11pt,a4paper]{article}

\usepackage{fancyhdr}
%\usepackage[T1]{fontenc}
\usepackage{amsfonts}
\usepackage{savetrees}




\begin{document}

\title{Homotopy co-momentum maps in Multisymplectic geometry.}
\author{Antonio Michele Miti\\ UCSC Brescia \& KU Leuven}

\date{December 18, 2019}
\maketitle


\begin{abstract}


% Introductory part, copied almost verbatim from the paper with Leo!
\emph{Multisymplectic structures} (also called \emph{``$n$-plectic''}) are a rather straightforward generalization of symplectic ones where closed non-degenerate 
($n$+$1$)-forms replace $2$-forms.
\\
Historically, the interest in multisymplectic manifolds, i.e. smooth manifolds equipped with an $n$-plectic structure,  has been motivated by
the need of understanding the geometrical foundations of first-order classical field theories.
%The key point is that, just as one can associate a symplectic manifold to an ordinary classical mechanical system (e.g. a single point-like particle constrained to some manifold), it is possible to associate a multisymplectic manifold to any classical field system (e.g. a continuous medium like a filament or a fluid).
%It is important to stress that mechanical systems are not the only source of inspiration for instances of this class of structures. For example, any oriented $n$-dimensional manifold can be considered $(n-1)$-plectic when equipped with a volume form and semisimple Lie groups have a natural interpretation as $2$-plectic manifolds.

As proposed by Rogers in \cite{Rogers2010} (see also \cite{Zambon2012}), this generalization can be expanded by introducing a higher analogue of the Poisson algebra of smooth functions (also known as ``\emph{observable algebra}'')  to the multisymplectic case.
Namely, Rogers proved that the algebraic structure encoding the observables on a multisymplectic manifold is the one of an $L_{\infty}$-algebra, that is, a graded vector space endowed with skew-symmetric multilinear brackets satisfying the Jacobi identity up to coherent homotopies.
\\
The latter structure allowed for a natural extension of the notion of moment map, called ``\emph{homotopy comoment map}'' (HCMM), originally defined in \cite{Callies2016}, associated to an infinitesimal action of a Lie group on a manifold preserving the multisymplectic form. Remarkably, to this moment map is also associated a suitable notion of conserved momenta \cite{Ryvkin2016}.
%
Being this concept particularly subtle and technical, finding new and meaningful examples represent an interesting and valuable task.

The goal of this presentation is to briefly review these ideas together with some new explicit constructions worked in full details in our research.
\\
% Spera Paper
A first result is the explicit expression of a HCMM pertaining to volume-preserving diffeomorphisms acting on oriented Riemannian manifolds. In the case of the standard Euclidean 3-dimensional spaces, this map transgresses to the standard hydrodynamical co-momentum map of Arnol’d, Marsden and Weinstein. It also enjoys interesting relations with the covariant phase space of ideal fluids (with linked vortices) and  (Massey) higher-order linking numbers.\cite{Miti2018} 
\\
% Leo Paper
Secondly, we present a complete classification for compact effective group actions on spheres together with some explicit constructions of HCMM in interesting particular cases.\cite{Miti2019} 
\\
%Zambon paper
At last,  we consider the higher analogue of the embedding of the 2-plectic algebra of observables into the corresponding twisted Courant algebroid
showing that such morphism satisfies a well-behaving condition under gauge-transformations when a HCMM exists.\cite{Mitia}

Overall, this is joint work with Mauro Spera, Leonid Ryvkin and Marco Zambon.



 
\end{abstract}




%------------------------------------------------------------------------------------------------
% Bibliography (BibTex)
% https://arxiv.org/hypertex/bibstyles/
%------------------------------------------------------------------------------------------------
			\nocite{*}
			\bibliographystyle{ieeetr}
			\bibliography{biblio}
%------------------------------------------------------------------------------------------------

\end{document}


